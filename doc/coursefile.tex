\PassOptionsToPackage{unicode=true}{hyperref} % options for packages loaded elsewhere
\PassOptionsToPackage{hyphens}{url}
%
\documentclass[
]{article}
\usepackage{lmodern}
\usepackage{amssymb,amsmath}
\usepackage{ifxetex,ifluatex}
\ifnum 0\ifxetex 1\fi\ifluatex 1\fi=0 % if pdftex
  \usepackage[T1]{fontenc}
  \usepackage[utf8]{inputenc}
  \usepackage{textcomp} % provides euro and other symbols
\else % if luatex or xelatex
  \usepackage{unicode-math}
  \defaultfontfeatures{Scale=MatchLowercase}
  \defaultfontfeatures[\rmfamily]{Ligatures=TeX,Scale=1}
\fi
% use upquote if available, for straight quotes in verbatim environments
\IfFileExists{upquote.sty}{\usepackage{upquote}}{}
\IfFileExists{microtype.sty}{% use microtype if available
  \usepackage[]{microtype}
  \UseMicrotypeSet[protrusion]{basicmath} % disable protrusion for tt fonts
}{}
\makeatletter
\@ifundefined{KOMAClassName}{% if non-KOMA class
  \IfFileExists{parskip.sty}{%
    \usepackage{parskip}
  }{% else
    \setlength{\parindent}{0pt}
    \setlength{\parskip}{6pt plus 2pt minus 1pt}}
}{% if KOMA class
  \KOMAoptions{parskip=half}}
\makeatother
\usepackage{xcolor}
\IfFileExists{xurl.sty}{\usepackage{xurl}}{} % add URL line breaks if available
\IfFileExists{bookmark.sty}{\usepackage{bookmark}}{\usepackage{hyperref}}
\hypersetup{
  pdfborder={0 0 0},
  breaklinks=true}
\urlstyle{same}  % don't use monospace font for urls
\setlength{\emergencystretch}{3em}  % prevent overfull lines
\providecommand{\tightlist}{%
  \setlength{\itemsep}{0pt}\setlength{\parskip}{0pt}}
\setcounter{secnumdepth}{-2}
% Redefines (sub)paragraphs to behave more like sections
\ifx\paragraph\undefined\else
  \let\oldparagraph\paragraph
  \renewcommand{\paragraph}[1]{\oldparagraph{#1}\mbox{}}
\fi
\ifx\subparagraph\undefined\else
  \let\oldsubparagraph\subparagraph
  \renewcommand{\subparagraph}[1]{\oldsubparagraph{#1}\mbox{}}
\fi

% set default figure placement to htbp
\makeatletter
\def\fps@figure{htbp}
\makeatother


\date{}

\begin{document}

\hypertarget{canvas-development---generating-course-files}{%
\section{Canvas Development - Generating Course
Files}\label{canvas-development---generating-course-files}}

This tool downloads material from your Canvas course site. Specifically,
it looks in the `Assignments', `Quizzes', and `Disussions' sections of
the course site.

\hypertarget{getting-started}{%
\subsection{Getting Started}\label{getting-started}}

These instructions will get you a copy of the project up and running on
your local machine for generating the course files.

\hypertarget{prerequisites}{%
\subsubsection{Prerequisites}\label{prerequisites}}

Functional requirements include Python 3.x and \texttt{pip} (Python
package manager) as well as Google Chrome (any recent version) installed
on your machine before hand.

These are the python packages you would need before running the script.

\begin{itemize}
\tightlist
\item
  canvasapi
\item
  webdriver-manager
\item
  selenium
\item
  selenium-wire
\end{itemize}

Download these modules by running one of the following commands on the
command line depending on your platform.

Windows: \texttt{pip\ install\ \textless{}package-name\textgreater{}}

*nix: \texttt{pip3\ install\ \textless{}package-name\textgreater{}}

\hypertarget{retrieving-the-access-token}{%
\subsubsection{Retrieving the Access
Token:}\label{retrieving-the-access-token}}

To run the python file \texttt{coursefiles.py} and generate course
files, a canvas access token is required.

\begin{enumerate}
\def\labelenumi{\arabic{enumi}.}
\tightlist
\item
  In Canvas, go to Account (top left) -\textgreater{} Settings.
\item
  On the Settings page, click on ``+ New Access Token'' button under the
  ``Approved Integrations'' section.
\item
  Enter the Purpose, ``Automatic Course File'', and leave the expiry
  date blank. Click on ``Generate Token''.
\item
  A token appears as a long string of characters in the next window.
  Copy it and paste it to a text editor. This has to be given to the
  program as an input.
\end{enumerate}

\hypertarget{additional-material}{%
\subsection{Additional Material}\label{additional-material}}

The tool can download certain folders from the Files section of the site
if they are named as follows.

\begin{enumerate}
\def\labelenumi{\arabic{enumi}.}
\tightlist
\item
  ``syllabus.pdf'': this file, if found, gets copied to the ``Course
  Syllabus'' sub-directory in the output folder
\item
  ``Slides/'': the contents of this folder, if found, are copied to the
  ``Lecture Notes'' sub-directory in the output folder
\item
  ``Assignments/'': the contents of this folder, if found, are copied to
  the ``Assessments and Sample solutions/Assignment Copies/''
  sub-directory in the output folder
\item
  ``Assignment Solutions/'': the contents of this folder, if found, are
  copied to the ``Assessments and Sample solutions/Model Assignment
  Solutions/'' sub-directory in the output folder
\end{enumerate}

\hypertarget{running-the-file}{%
\subsection{Running the file}\label{running-the-file}}

Before running the program, ensure that you have a strong and stable
Internet connection. Run the \texttt{coursefiles.py} python file from
the command line as follows depending on your platform.

Windows: \texttt{python\ coursefiles.py}

*nix: \texttt{python3\ coursefiles.py}

You will then be asked for certain inputs: 1. Your access token: This is
the token that you generated above. Please copy it here from your text
editor where you had pasted it. 1. Your course url: This is the URL of
the Canvas site of the course whose course file you want to generate. It
is of the form: https://hulms.instructure.com/courses/. You can
copy-paste it from your browser.

A new Google Chrome window will now open and you will be redirected to
the the Habib University's main LMS page. Please log in here to initiate
the download process.

AS the program runs, various pages will automatically load and get
downloaded in this Google Chrome window. You can safely ignore the
winow, but take care to \emph{NOT} close it yet. The console from where
you ran the command will show the progress. The download takes some time
and the Google Chrome window closes automatically once the download is
complete. You can now find the ``coursefiles'' folder of the same name
as your course in the working directory.

\hypertarget{important-instructions}{%
\subsection{Important Instructions}\label{important-instructions}}

\begin{enumerate}
\def\labelenumi{\arabic{enumi}.}
\tightlist
\item
  Do not close the browser window during execution. It might come to the
  foreground repeatedly due to page reloads. If som dock it to the
  background.
\item
  The tool only works for an ongoing course.
\end{enumerate}

\hypertarget{error-instructions}{%
\subsection{Error Instructions}\label{error-instructions}}

Any error during the download process can be identified through the
following:

\begin{enumerate}
\def\labelenumi{\arabic{enumi}.}
\tightlist
\item
  The console gives some error message instead of the usual progress
  logs.
\item
  The browser window becomes idle for a very long time. This error will
  be reflected in the console.
\item
  The produced folder has missing items.
\item
  The pdfs are generated incorrectly.
\end{enumerate}

Kindly take a screenshot of the console error message on the console, a
screenshot of the browser window and a breif error description and mail
it to \texttt{fr06161@st.habib.edu.pk}.

\hypertarget{code}{%
\section{Code}\label{code}}

The entire code is available in the file \texttt{coursefiles.py} which
you have to run. You can find the latest version online at:
https://github.com/Fizza-Rubab/Canvas-Developer . Contributions are
welcome!

\end{document}

%%% Local Variables:
%%% mode: latex
%%% TeX-master: t
%%% End:
